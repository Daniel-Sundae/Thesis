\chapter{Introduction}

In a recent press release, Gartner estimated cloud computing to reach a market size of about 600 billion dollars in the year of 2023 \cite{Gartner2023}. It is clear that the need for delegated storage and delegated computing has grown steadily as data has become more valuable. From a security perspective, delegated computation is a challenging problem. Normally, the client encrypts the data on their secure local machine and sends it to the server. The server decrypts and performs computations on the raw, unencrypted data, then re-encrypts the result and sends it back to the client, who decrypts it. For this protocol to work, the server needs access to the secret key for decrypting the data, meaning having access to the raw data. However, this raises privacy concerns. What if the server is malicious and leaks the data for profit? What if the server is compromised? To protect the client, the computation should ideally not require any information about the raw data. More generally, is there a safe way to allow for any third party to process sensitive data? How does one compute on encrypted data without first decrypting it?

Homomorphic encryption is an encryption method that allows for computation on encrypted data. To compute on ciphertexts, homomorphic encryption schemes need to satisfy what is called a homomorphic property; an operation on ciphertexts in the encrypted domain corresponds to an operation on the plaintexts in the unencrypted domain. Some encryption schemes have one natural homomorphic property, such as the multiplication operation in the RSA cryptosystem. However, to allow for general computation on ciphertexts, the scheme in question need to satisfy two types of operations; multiplication and addition of ciphertexts. It turns out that any computable function can be represented as a combination of these two operations. The question of secure, arbitrary computation on encrypted data therefore boils down to the following quesiton; does there exist a secure encryption scheme that satisfies an arbitrary (possibly infinite) combination of addition and multiplication operations on ciphertexts? 

A scheme that satisfies the above is called a fully homomorphic encryption scheme. The problem of finding one has been dubbed as the "holy grail of cryptography" and remained an open problem for more than 30 years. In 2009, Craig Gentry \cite{Gentry-Thesis}, solved this problem. Gentry showed the existence of a secure encryption scheme that satisfies arbitrary multiplication and addition of ciphertext. His construction was complicated and inefficient, but it launched the development of more efficient schemes. Many research grants have been provided towards the development of more efficient schemes. The applications areas of homomorphic encryption essentially includes everything that processes sensitive information. Some examples of areas where computation on encrypted data is particularly relevant are healthcare records, machine learning, genome sequencing, navigation, electronic voting and financial records.