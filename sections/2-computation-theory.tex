\chapter{Computation theory}
 In this section, it is assumed that the reader has prior knowledge of deterministic- and probabilistic Turing machines as a model of computation (an excellent introduction can be found in \cite{Gol01}). We introduce an alternative model of computation based on sets of circuits for the purpose of protection against stronger adversaries. We include the basics needed to understand cryptography and homomorphic encryption in particular. 

\section{Digital logic}

\begin{definition}[Circuit]
For $n, m \in \mathbb{N}$ and any field $(\mathbb{F},+,\times)$, an arithmetic circuit is a vector-valued polynomial function $C \colon \mathbb{F}^{n} \to \mathbb{F}^m$. 
\end{definition}

A circuit $C$ is represented by a finite directed acyclic graph with $n$ source nodes (the $n$ inputs) and $m$ sink nodes (the $m$ outputs). The internal nodes of the circuit are called \emph{gates} and are stacked in layers. For more details about the structure of a circuit, see \cite{goldreich_2008} or \cite{MF21}. The number of nodes in $C$ is called its \emph{size} and is denoted $|C|$. The longest path in $C$ is called its \emph{depth}.
A circuit is called \emph{Boolean} when the field is $\mathbb{F}_2$ and each gate takes at most 2 inputs. Boolean circuits and arithmetic circuits are equivalent in the sense that the set of functions that can be computed by an arithmetic circuit is equal to the set of functions that can be computed by a Boolean circuit.\footnote{A Boolean circuit is an arithmetic circuit. Conversely, any arithmetic circuit can be simulated by representing the inputs and outputs as a bitstring and utilizing that XOR and AND is a complete set of gates.} If the input is a string of bits, we assume the circuit is Boolean.

We consider circuits as algorithms and use them as an alternative approach to the traditional Turing machine model of computation.\footnote{The reason for this alternative model is to assume adversaries are computationally "stronger". See Theorem \ref{thm:compl-class}.} Notice that any given circuit, $C$, can only compute on inputs of the same length whereas a Turing machine $M$ takes inputs of any size $n$. However, a circuit always halts on a given input whereas a Turing machine may not. For the purpose of our discussion relating to cryptography, we assume every Turing machines halts unless otherwise stated. To allow circuits to handle arbitrary length inputs we consider families of circuits.  

\begin{definition}[Circuit family \cite{MF21}]
A circuit family $C = \{C_n\}_{n \in \mathbb{N}}$ is an indexed set of circuits $C_n \colon \mathbb{F}^{n + r} \to \mathbb{F}^m$ where $r,m = \operatorname{poly}(n)$.
\end{definition}

For any input $x$ with length $n$, $C(x) \stackrel{\mathrm{def}}{=} C_n(x)$. For each circuit $C_n \in C$, $r$ represents the random coins used. If $r = 0$ for all $n$ then $C$ is a deterministic circuit family. A circuit family is said to have polynomial-size if there exists a polynomial $p$ such that $|C_n| < p(n)$ for all $n$. 

\section{Complexity classes}\label{subsec:Complexity classes}
\begin{definition}[Complexity Class $\operatorname{P}$]
$\operatorname{P}$ is the set of languages $\mathscr{L}$ such that there exists a deterministic polynomial-time Turing machine $M$ satisfying $M(x) = 1 \iff x \in \mathscr{L}$ 
\end{definition}
\begin{definition}[Complexity Class $\operatorname{BPP}$]
$\operatorname{BPP}$ is the set of languages $\mathscr{L}$ such that there exists a probabilistic polynomial-time (PPT) Turing machine $M$ satisfying
\begin{align*}
& \operatorname{Pr}[M(x)=1] \geq 2/3 \text{ if $x \in L$}
\\
& \operatorname{Pr}[M(x)=1] \leq 1/3 \text{ if $x \notin L$}
\end{align*}
\end{definition}
\begin{definition}[Complexity Class $\operatorname{P/poly}$]
$\operatorname{P/poly}$ is the set of languages $\mathscr{L}$ such that there exists a polynomial-size circuit family $C$ satisfying $C(x) = 1 \iff x \in \mathscr{L}$ 
\end{definition}
Informally speaking, circuit families is a stronger model of computation than the PPT Turing machine model in the sense that if there exists a PPT Turing machine for deciding a problem, then there also exists a circuit family that can decide the same problem. The formal statement is as follows: 
\begin{theorem}
    \label{thm:compl-class}
    $\operatorname{P} \subseteq \operatorname{BPP} \subsetneq \operatorname{P/poly}$
\end{theorem}
The first inclusion follows from the fact that if there exists a deterministic Turing machine that decides a language, then that same machine can be seen as a PPT machine that ignores a given input sequence of coin tosses. For the second inclusion, consider a language $\mathscr{L} \in \operatorname{BPP}$ and a corresponding PPT machine $M$ for $\mathscr{L}$. Then, for any given input $x_n$ with length $n$, at least $2/3$ of the set of all possible coin toss sequences are good (good $r$ means $M(x_n;r) = 1 \iff x_n \in \mathscr{L}$). This means that there exists at least 1 sequence of coin tosses that yields the correct result for $2/3$ of the possible inputs of length $n$. Consider machine $M'$ that on input $x_n$ runs $M(x_n)$ many (still $\operatorname{poly}(n)$) times and outputs the majority result. Then, the error probability vanishes exponentially, meaning there are exponentially few coin toss sequences that are bad for $M'$ (see \cite{Arora,Gol01} for more detail). Therefore there exist a coin toss sequence, $r_n$, that yields the correct result for all inputs of length $n$. Consider circuit $C_n \colon \{0,1\}^{n+|r_n|} \to \{0,1\}$ with $r_n$ hardcoded as inputs where $C_n$ simulates $M'$ using $r_n$, i.e., $C_n(x_n) = M'(x;r)$. Therefore $C_n(x) = 1 \iff x \in \mathscr{L}$ and $C$ decides $\mathscr{L}$.

Interestingly the first inclusion is speculated to be set equivalence \cite[pp. 126]{Arora}, meaning that a deterministic machine could decide the same languages as a probabilistic one. The second inclusion is proper since every unary language is in $\operatorname{P/poly}$ whereas undecidable ones are not in $\operatorname{BPP}$ (see \cite[pp. 110]{Arora} for details). In this sense, circuit families are a stronger model of computation than PPT Turing machines. We capture this notion with uniformity. 

\begin{definition}[Uniform circuit family]
A circuit family $\{C_n\}_{n \in \mathbb{N}}$ is uniform if there exists a polynomial-time Turing machine $M$ such that $M(1^n)$ outputs the description of $C_n$ for all $n\in \mathbb{N}$.
\end{definition}

A uniform circuit family is polynomial size. The converse is not necessarily true. A family that is not uniform is said to be a non-uniform circuit family. Note that Turing machines are at least as strong as uniform circuit families. More formally, if a uniform circuit family decides $\mathscr{L}$ then there exists a polynomial-time Turing machine that decides $\mathscr{L}$.\footnote{The converse is also true, meaning deterministic polynomial-time Turing machines are exactly as powerful as uniform circuit families. See \cite[pp. 111]{Arora} for details} Simply construct the polynomial-time Turing machine that given any input $x \in \mathscr{L}$, first generates a description of $C_{|x|}$ and then simulates $C_{|x|}(x)$. In other words, the non-uniform circuit families are stronger than the polynomial-time Turing machines.
